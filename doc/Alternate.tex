\section*{\centering Assignment 7}
\vspace{15px}
Write a program which takes n integers as input(max 50 integers)
and stores them in an array data from index 0 to n-1. Now we want to 
rearrange the integers in the following way:- find the minmum
value and put it in position (n/2) [for odd number of elements]
oand position (n/2-1) [for even number number elements]) ;
find the second smallest value and put it to its right;
then the third small and place it to its lieft and so 
on altering right and left until all the integers are done.\\\\
Initial array:- (size: 7)\\
\begin{tabular}{| c | c | c | c | c | c |}
	\hline
	7 & 3 & 1 & 6 & 4 & 2 \\
	\hline
\end{tabular}\\
Intial array:- (size: 6)\\
\begin{tabular}{| c | c | c | c | c | c | c |}
	\hline
	7 & 3 & 1 & 6 & 4 & 2 & 5 \\
	\hline
\end{tabular}\\
After re-arrangement of the first array becomes\\
\begin{tabular}{| c | c | c | c | c | c |}
	\hline
	6 & 3 & 1 & 2 & 4 & 7 \\
	\hline
\end{tabular}\\
After re-arrangement the second array becomes\\
\begin{tabular}{| c | c | c | c | c | c | c |}
	\hline
	7 & 5 & 3 & 1 & 2 & 4 & 6 \\
	\hline
\end{tabular}

\section*{Algorithm:}
\textbf{\color{javapurple}Class Avarage\_main:}\\
\textbf{Method Main:}\\
Step 1: create a input handle to accept input from the user\\
Step 2: take the input of number of students in N \\
Step 3: Check if the input makes sense if not exit \\
Step 4: create temporary and call displaySmall\\\\
\textbf{\color{javapurple}Class Avarage:}\\
\textbf{Method displaySmall:}\\
Step 1: create 2 values avgl and namel to store the name and the avarage of least student\\
Step 2: start a loop and continue looping until N is 0.\\
Step 3: take input of 5 subjects from the stdin.\\
Step 4: create a temporary variable and store the avarage of the 5 subjects.\\
Step 5: check if the avarage of tmp variable is lower than the avgl.\\
Step 6: if the check passes then replace the value of avgl with avg. \\
Step 7: and also replace the name depending on Step 5 condition. \\
Step 8: reduce N by 1.\\
Step 9: after completing the whole loop print the results.\\\\

\section*{Source code:}
\lstinputlisting{src/Alternate.java}
\lstinputlisting{src/Alternate_main.java}

\section*{Variable Listing:}
\begin{center}
\begin{tabular}{ | c | l | c | l |}
\hline
	\multicolumn{1}{ | c | }{\textbf{Name}} & 
	\multicolumn{1}{ | c | }{\textbf{Function}} &
	\multicolumn{1}{ | c | }{\textbf{Type}} &
	\multicolumn{1}{ | c | }{\textbf{Scope}} \\
\hline
	arr & the input array which gets sorted & int[] & fill,main \\ 
	nx & the buffer that is to be filled & int[] & fill,main \\
	pos & the starting position of the filler & int & fill\\
	i, j & loop control variable of the sorter & int & fill, main\\
	tmp & temporary variable for swapping & int & fill\\
	n & number of elements in the array & int & main\\
\hline
\end{tabular}
\end{center}
