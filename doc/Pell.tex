\section*{\centering Assignment 2}
\vspace{15px}
Write a Java Program to pring the first N numbers of the Pell series.\\
In mathematics, the Pell numbers are an infinite sequence of integers. The sequence of Pell 
numbers starts with 0 and 1, and then each Pell numbers is the sum of twice the previous Pell number 
and the Pell number before that.:\\\\
thus, 70 is the companion to 29, and 70 = 2 * 29 + 12 = 58 + 12.\\\\
The first few terms of the sequence are :\\\\
0, 1, 2, 5, 12, 29, k70, 169, 408, 985, 2375, 5741, 1386

\section*{Algorithm:}
\textbf{\color{javapurple}Class Pell\_main:}\\
\textbf{Method Main:}\\
Step 1: create a input handler and take the number of terms as input \\
Step 2: if the terms have a less number than 0 then promp user for reinput \\
Step 3: create a object of Pell class and call display method\\\\
\textbf{\color{javapurple}Class Pell:}\\
\textbf{Method display:}\\
Step 1: create two variable for first and secon pell numbers\\
Step 2: start a `for' loop from 0 to N with the loop control i\\
Step 3: print 2 * second term + first term to the screen and also store it in c\\
Step 4: set first term to second term and second term to c.\\\\

\section*{Source code:}
\lstinputlisting{src/Pell.java}
\lstinputlisting{src/Pell_main.java}

\section*{Variable Listing:}
\begin{center}
\begin{tabular}{ | c | l | c | l |}
\hline
	\multicolumn{1}{ | c | }{\textbf{Name}} & 
	\multicolumn{1}{ | c | }{\textbf{Function}} &
	\multicolumn{1}{ | c | }{\textbf{Type}} &
	\multicolumn{1}{ | c | }{\textbf{Scope}} \\
\hline
	sc & input handler object that is used for input & Scanner & main() \\
	N & number of pell series terms & int & main(),display() \\
	a & first term of pell series  & int & display() \\
	b & second term of pell series & int & display() \\
	c & temporary variable that is used for storing tmp value of stuff & int & display() \\
	px & Pell Class object used to call display() & Pell & main() \\
\hline
\end{tabular}
\end{center}
