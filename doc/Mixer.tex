\section*{\centering Assignment 6}
\vspace{15px}
A class Mixer has been defined to merge two sorted integer array in ascending 
order. Some of the members of the class are given below: \\\\
\begin{tabular}{l l}
	\textbf{Class Name:} & \textbf{Mixer} \\\\
	\textbf{Data members/instance variables:} & \\
	int arr[] & : to store elements of an array \\
	int n & : to store the size of the array \\\\
	\textbf{Member functions:} & \\
	Mixer(int nn) & : constructor to assign n = nn \\ 
	void accept() & : to accept the elements of the array in\\
	&ascending order without any duplicates \\
	Mixer mix(Mixer A) & : to merge the current object array\\
	&elements with the parameterized array elements \\
	&and return the rusultant object.\\
	void display() & : to display elements of the array \\\\

\end{tabular}

\section*{Algorithm:}
\textbf{\color{javapurple}Class Pell\_main:}\\
\textbf{Method Main:}\\
Step 1: create a input handler using Scanner calss.\\
Step 2: create 2 variables to hold the length of the mixer arrays\\
Step 3: check if the input is sensible and greatere than 0\\
Step 4: if the check fails propmt the user again to input number\\
Step 5: create two objects of mixer class using the previously declared variables.\\
Step 6: accept input from the user using the accept() method.\\
Step 7: call the mix method using the mix objects that are created.\\
Stpe 8: display the mixed arrays.\\\\
\textbf{\color{javapurple}Class Mixer:}\\
\textbf{Method Mixer:}\\
Step 1: set the object variable n to nn \\
Step 2: create a array of size n and store it in object variable to arr\\\\
\textbf{Method mix:}\\
Step 1: create a new object for mixer with the size of the two input mixer arrays.\\
Step 2: fill the newly created array with the elements of A and this.\\
Step 3: sort the array of the newly created object using insertion sort.\\
Step 4: return the object.\\\\
\textbf{Method accept:}\\
Step 1: create a input handler using Scanner class.\\
Step 2: loop throught the arr array of the current object and fill it with input.\\\\
\textbf{Method display:}\\
Step 1: loop througth the whole arr array of current object and print the elements.\\
Step 2: print a newline for asthetics.\\\\

\section*{Source code:}
\lstinputlisting{src/Mixer.java}
\lstinputlisting{src/Mixer_main.java}

\section*{Variable Listing:}
\begin{center}
\begin{tabular}{ | c | l | c | l |}
\hline
	\multicolumn{1}{ | c | }{\textbf{Name}} & 
	\multicolumn{1}{ | c | }{\textbf{Function}} &
	\multicolumn{1}{ | c | }{\textbf{Type}} &
	\multicolumn{1}{ | c | }{\textbf{Scope}} \\
\hline
	n1 & number of elements & int & main \\
	n2 & number of elements & int & main \\
	m1 & object to be filled & Mixer & main \\
	m2 & object to be filled & Mixer & main \\
	m3 & output of the mixers & Mixer & main \\
	arr & to store elements of an array & int[] & Mixer \\
	n & to store size of arr & int & Mixer \\
	nn & temporary input var & int & Mixer() \\
	sc & input of all the numbers & Scanner & accept, main\\
	A & mix object to be used to mix stuff & Mixer & mix \\
	m & output of the mix method & Mixer & mix \\
	i,j & loop control variables & int & mix \\
	key & key of insertion sort & int & mix \\
\hline
\end{tabular}
\end{center}
