\section*{\centering Assignment 10}
\vspace{15px}
A company manufactures packingg cartons in four sizes, i.e. cartons to 
accommodate 6 boxes, 12 boxes, 24 boxes and 48 boxes. Desiggn a program to 
accept the number of boxes to be packed (N) by the user (maximum up to 
1000 boxes) and display the break-up of the cartons used in descending order
of capacity (i.e. preference should be given to the highest capacity
available, adn if boxes left are less than 6, an extra carton of capacity
6 should be used.)\\\\
Test your program with the following data and some random data:\\\\
\textbf{Example 1:}\\\\
\textbf{INPUT:}	N = 726\\
\textbf{OUTPUT:}	48 x 15 = 720\\
			6 x 1 = 6\\
	Remaining boxes	= 0\\
	Total number of boxes	= 726\\
	Toatl number of cartons = 16\\\\
\textbf{Example 2:}\\\\
\textbf{INPUT:}	N = 726\\
\textbf{OUTPUT:}	48 x 15 = 720\\
			6 x 1 = 6\\
	Remaining boxes	= 0\\
	Total number of boxes	= 726\\
	Toatl number of cartons = 16\\

\section*{Algorithm:}
\textbf{\color{javapurple}Class Company\_main:}\\
\textbf{Method Main:}\\
Step 1: create a input handle using Scanner class\\
Step 2: take input the number of boxes.\\
Step 3: check if the input is sane. if the input is not sane print INVALID INPUT.\\
Step 4: create a compmany object using the N as intput.\\ 
Step 5: call calculate using company object.\\
Step 6: call pretty\_print using company object.\\\\
\textbf{\color{javapurple}Class Company:}\\
\textbf{Method calculate:}\\
Step 1: create a copy of N locally.\\
Step 2: loop from 0 to this.cartons.length using the variable i\\
Step 3: execute the statement num\_cartons[i] = N / cartons[i]\\
Step 4: execute the statement num\_carton++\\
Step 5: execute the statement N \%= cartons[i]\\
Step 6: set remainder to N\\\\
\textbf{Method pretty\_print:}\\
Step 1: loop from 0 to this.num\_cartons.lengths using the variable i\\
Step 2: execute print statement only if num\_cartons[i] not equals 0\\
Step 3: print the output in a fancy format.\\\\
\textbf{Method Company:}\\
Step 1: initialize cartons with [48 24 12 6] for carton listings.\\
Step 2: initialize num\_cartons with a new buffer of number of type of cartons\\
Step 4: initialize N with input N.\\
Step 5: initialize num\_carton to zero\\

\section*{Source code:}
\lstinputlisting{src/Company.java}
\lstinputlisting{src/Company_main.java}

\section*{Variable Listing:}
\begin{center}
\begin{tabular}{ | c | l | c | l |}
\hline
	\multicolumn{1}{ | c | }{\textbf{Name}} & 
	\multicolumn{1}{ | c | }{\textbf{Function}} &
	\multicolumn{1}{ | c | }{\textbf{Type}} &
	\multicolumn{1}{ | c | }{\textbf{Scope}} \\
\hline
	sc & input handler object that is used for input & Scanner & main \\
	N & number of box & int & main,Company:obj,calculate \\
	c & Company obbject createor & Company & main\\
	i,j & iterator control variable & pretty\_print & pretty\_print, calculate\\
	cartons & the varid capacity boxes list & int[] & Company:obj\\ 
	num\_cartons & the magnitude of each boxes & int[] & Company:obj\\ 
	num\_carton & this is the total number of cartons that are required & int & Company:obj\\
	remainder & the remaining boxes after filling & int & Company:obj\\
\hline
\end{tabular}
\end{center}
