\section*{\centering Assignment 8}
\vspace{15px}
Two matrices are said to be equal if they have the same dimention and their
corresponding elements are equal.\\
For examples, the two matrix A and B given below are equal:
\begin{center}
\begin{tabular}{ l r }
	Matrix A & Matrix B \\
	\begin{tabular}{|c|c|c|}
\hline
		1 & 2 & 3 \\ \hline
		2 & 4 & 5 \\ \hline
		3 & 5 & 6 \\ \hline
	\end{tabular} & 
	\begin{tabular}{|c|c|c|}
\hline
		1 & 2 & 3 \\ \hline
		2 & 4 & 5 \\ \hline
		3 & 5 & 6 \\ \hline
	\end{tabular}\\
\end{tabular}
\end{center}
Design of class EqMat to check if two matrices are equal or not.
Assume that the two matrices have the same dimention.
some of the membersof the class are given below:\\
\begin{tabular}{ll}
	\textbf{Class name} & \textbf{EqMat} \\
	\textbf{Data  members/instance}\\
	\textbf{Members:}\\
	a[][] & to store integer elements\\
	m & to store number of rows \\
	n & to store number of columns\\
	\textbf{Members functions\ methods:}\\
	EqMat(int mm, int nn) & parameterized constructor initialize the data\\
	& members m = mm and n = nn ;\\
	void readarray() & to enter elemtents in the array \\
	int check(EqMat p, EqMat q) & checks if the parametarized objects p and q\\
	& are equal and returns 1 if true otherwise returns 0. \\
	void print() & displays the arrays elemtents. \\
\end{tabular}
Define the class EqMat giving details of constructor(), void readarray(), int check(EqMat, EqMat) and void
print(). Define the main() function to create objects and call the funcitons accordingly to enable the task.

\section*{Algorithm:}
\textbf{\color{javapurple}Class EqMat\_main:}\\
\textbf{Method main:}\\
Step 1: take input from the user about the row and cols.\\
Step 2: check if the input is greater than 0 and retake input if not.\\
Step 3: create a EqMat object and call readarray for input array.\\
Step 4: repeat the last three steps again for a new object.\\
Step 5: call print function of the first object.\\
Step 6: call print function of the second object.\\
Step 7: check if a.check(b) is equal to 1\\
Step 8: if the check passes print array are equal\\
Step 9: if the check fails print array are not equal\\\\
\textbf{\color{javapurple}Class EqMat:}\\
\textbf{Method Eqmat:}\\
Step 1: initialize all object variables\\
Step 2: create a new object for object a variable\\\\
\textbf{Method check:}\\
Step 1: check if the matrix are even same dimentional.\\
Step 2: if the check fails return 0 ; \\
Step 3: loop from 0 to p.m using i as variable\\
Step 4: loop from 0 to q.n using j as variable \\
Step 5: check if p.a[i][j] is not equal to q.a[i][j]. \\
Step 6: if the check fails return 0 ;\\
Step 7: return if the control makes it out of the loops.\\\\
\textbf{Method print:}\\
Step 1: loop from 0 through the number of rows using i as var.\\
Step 2: loop from 0 through the number of cols using j as var.\\
Step 3: print a[i][j] with a space at the end.\\
Step 4: print a new line at the end of j var.\\\\
\textbf{Method readarray:}\\
Step 1: create a input stream handle.\\
Step 2: take input in the buffer created in a.\\\\

\section*{Source code:}
\lstinputlisting{src/EqMat.java}
\lstinputlisting{src/EqMat_main.java}

\section*{Variable Listing:}
\begin{center}
\begin{tabular}{ | c | l | c | l |}
\hline
	\multicolumn{1}{ | c | }{\textbf{Name}} & 
	\multicolumn{1}{ | c | }{\textbf{Function}} &
	\multicolumn{1}{ | c | }{\textbf{Type}} &
	\multicolumn{1}{ | c | }{\textbf{Scope}} \\
\hline
	a[][] & integer matrix storer & int[][] & EqMat:object\\
	m & number of rows & int & EqMat:object, main \\
	n & number of cols & int & EqMat:object, main \\
	mm & a temporary variable for m & int & EqMat \\
	nn & a temporary variable for n & int & EqMat \\
	p & input for check function & EqMat & check \\
	q & input for check function & EqMat & check \\
	i & loop control varible & int & print \\
	j & loop control varible & int & print \\
	sc & Input handle varable & Scanner & readarray, main \\
	a,b & object for testing & main & main \\
\hline
\end{tabular}
\end{center}
